\documentclass{article}

\title{Assignment:\\
Verification of an Internet Game Server}

\begin{document}

\maketitle

\section{Assignment}
The goal of this assignment is to verify a game server. More specifically, you must add annotations to \texttt{GameServer.c} such that the VeriFast program verifier accepts the program. Modifying the source code itself or using \texttt{assume} statements is not permitted. It is not necessary for specifications to be complete; it suffices to prove the absence of memory errors and data races.

Your final solution must be submitted to \texttt{bart.jacobs@cs.kuleuven.be} by March 7th 2010. It suffices to send the annotated version of \texttt{GameServer.c}. If you get stuck, you can request hints by sending an email describing your problem to \texttt{bart.jacobs@cs.kuleuven.be}.

\section{Extra}
It is not necessary to complete the optional (but harder!) items in the list below to succeed for this assignment. However, completing them will be rewarded.

\begin{itemize}
  \item Verification of the function \texttt{create\_game\_last} is optional.
  \item It suffices to prove memory safety. However, verifying full functional correctness can be done.
\end{itemize}

\section{Hints}

\begin{itemize}
  \item To verify memory safety of pointer dereference \texttt{*p}, VeriFast looks for a heap chunk \texttt{pointer(p, \_)}. \texttt{prelude.h} contains various lemmas to create and manipulate \texttt{pointer} chunks.
\end{itemize}

\section{Compiling and Running the Server}

\subsection{Windows}

\subsubsection*{Compiling}
To compile the server on Windows, the Microsoft C/C++ compiler (\texttt{cl.exe}) must be installed. This compiler comes with visual studio. Enter the following command on the command-line:\newline\newline
\texttt{cl.exe /D "WIN32" *.c ws2\_32.lib}\newline\newline
The program can also be compiled in Visual Studio by opening GameServer.sln.
\subsubsection*{Running}
To run the server, run the executable generated by the compiler. Clients can connect via telnet\footnote{Telnet is not installed by default on Windows Vista and Windows 7. Go to \texttt{Control Panel->Programs and Features->Turn Windows Features on or off to install telnet.}} on port 1234:
\newline\newline
\texttt{telnet localhost 1234}

\subsection{Linux}

\subsubsection*{Compiling}
To compile the server on Linux, the GNU C compiler (\texttt{gcc}) must be installed. Enter the following command on the command-line:\newline\newline
\texttt{gcc todo}
\subsubsection*{Running}
To run the server, run the executable generated by the compiler. Clients can connect via telnet on port 1234:\newline\newline
\texttt{telnet localhost 1234}

\end{document}